\documentclass{beamer}
\usetheme{Madrid}

\usepackage{listings}
\lstset{language=[ISO]C++,
        keywordstyle=\color{blue},
        stringstyle=\color{red},
        commentstyle=\color{green},
        frame=single,
        }
        
\usepackage{lmodern}

\usepackage{color}

\author{David Schneider}

\setbeameroption{show notes}


\title{Object Oriented Design - Basics}

\begin{document}

\begin{frame}
\titlepage
\end{frame}

\begin{frame}{Introduction}
This is the introduction.
\end{frame}


\begin{frame}
Design is all about changable code.
Code is written once, but read many times.
\end{frame}

\section{Features}

\begin{frame}
\subsection{Objects and classes}
accessibility
\end{frame}

\subsection{Encapsulation}
\begin{frame}{Encapsulation}
abstraction
information hidding
hide internal details, which allows changing this details
\end{frame}

\subsection{Inheritance}
\begin{frame}{Inheritance}
Base class
Abstract class
\end{frame}

\subsection{Composition and delegation}
\subsection{Polymorphism}
\begin{frame}{Polymorphism}
subtyping
\end{frame}

\subsection{Open recursion}
\begin{frame}{Open recursion}[fragil]
Virtual methods

\lstinputlisting[caption=Open recursion]{lst/open_recursion.cpp}


\end{frame}


\section{Pattern}


\subsection{GRASP}

\subsection{SOLID}

\begin{frame}{S.O.L.I.D - Principals}
From by Robert C. Martin (Oncle Bob)


\begin{description}
\item [S] Single-responsiblity principle
\item [O] Open-closed principle
\item [L] Liskov substitution principle
\item [I] Interface segregation principle
\item [D] Dependency Inversion Principle
\end{description}
\end{frame}

\begin{frame}{Single-responsiblity principle}
a class should have only a single responsibility (i.e. only one potential change
in the software's specification should be able to affect the specification of the class)
\end{frame}

\begin{frame}{Open-closed principle}
''A module should be open for extension but closed for modification''
\end{frame}

\begin{frame}{Liskov substitution principle}
''objects in a program should be replaceable with instances of their subtypes
without altering the correctness of that program.''
\end{frame}

\begin{frame}{Interface segregation principle}
\end{frame}

\begin{frame}{Dependency Inversion Principle}
\end{frame}

\section{Clean Code}



\end{document}
